\documentclass[10pt,letterpaper]{article}

\usepackage{hyperref}
\usepackage{geometry}
\usepackage{enumitem}
\usepackage{multicol}
\newcommand{\tabitem}{~~\llap{\textbullet}~~}

% Fonts
\usepackage[T1]{fontenc}
\usepackage[urw-garamond]{mathdesign}

% Set your name here
\def\name{Samuel Perry}

% The following metadata will show up in the PDF properties
\hypersetup{
  colorlinks = true,
  urlcolor = black,
  pdfauthor = {\name},
  pdfkeywords = {DSP, Programmer},
  pdftitle = {\name: Curriculum Vitae},
  pdfsubject = {Curriculum Vitae},
  pdfpagemode = UseNone
}

\geometry{
  body={6.5in, 9.0in},
  left=1.0in,
  top=1.0in
}

% Customize page headers
\pagestyle{myheadings}
\markright{\name}
\thispagestyle{empty}

% Custom section fonts
\usepackage{sectsty}
\sectionfont{\rmfamily\mdseries\Large}
\subsectionfont{\rmfamily\mdseries\itshape\large}

% Other possible font commands include:
% \ttfamily for teletype,
% \sffamily for sans serif,
% \bfseries for bold,
% \scshape for small caps,
% \normalsize, \large, \Large, \LARGE sizes.

% Don't indent paragraphs.
\setlength\parindent{0em}

% Make lists without bullets and compact spacing
\renewenvironment{itemize}{
  \begin{list}{}{
    \setlength{\leftmargin}{1.5em}
    \setlength{\itemsep}{0.25em}
    \setlength{\parskip}{0pt}
    \setlength{\parsep}{0.25em}
  }
}{
  \end{list}
}
\setlist[enumerate]{itemsep=0.25em}

\begin{document}

% Place name at left
{\huge \name}

% Alternatively, print name centered and bold:
%\centerline{\huge \bf \name}

\bigskip

\begin{minipage}[t]{0.495\textwidth}
  20 Lower Luton Road\\
  Wheathampstead\\
  Hertfordshire\\
  AL4 8QZ\\

\end{minipage}
\begin{minipage}[t]{0.495\textwidth}
    Phone: (+44) 7766 521596\\
    Email: \href{mailto:samuel.perry89@gmail.com}{samuel.perry89@gmail.com} \\
    Linked-in: \\\href{https://uk.linkedin.com/in/sam-perry-04245438}{https://uk.linkedin.com/in/sam-perry-04245438}
\end{minipage}

\section*{Personal Profile}

A highly motivated and ambitious graduate with a background in programming and
digital signal processing in a musical context. Aiming to further knowledge and
understanding of digital signal processing techniques, building on previous
experience in this area. Capable of understanding and utilising signal
processing techniques for the realization of signal processing applications for
a variety of use cases, as proved through recent studies and employment.
Through further studies in a technically oriented environment,  the objective
is to gain a deeper understanding in this field in order to facilitate future
employment or research opportunities.

\section*{Employment}

\begin{itemize}
    \item Institut de Recherche et Coordination Acoustique/Musique (IRCAM)
\end{itemize}

\subsection*{IRCAM}
    Role: Student Research Assistant \\
    Team: Analysis \& Synthesis team \\
    Location: Paris, France \\
    Period: August 2014 - July 2015 \\
    \newline
    Description: \\
    Worked on a range of DSP related projects and tasks for the Analysis and
    Synthesis team. Modified and improved a number of programs, primarily in
    Python, with particular focus on vocal and musical processing.  Major
    project involved using audio descriptor analyses to drive transformations
    on vocal corpus. Worked alongside a variety of researchers and
    professionals developing new and innovative signal processing techniques in
    fields of research such as such as vocal transformations and audio/musical
    content analysis. \\
    \newline
    Key areas explored:
    \begin{itemize}
        \item \textit{Audio content analysis}\\
            Utilised a number of audio descriptors to test for similarities in
            audio for a content matching algorithm
        \item \textit{Vocal segmentation/classification}\\
            Improved the efficiency of the content matching algorithm through
            addition of vocal segment classification and tree search algorithm.
        \item \textit{Distributed computing/Asynchronous processing}\\
            Debugged and improved a program that utilised distributed task
            scheduling for computation heavy analysis of audio
    \end{itemize}

\newpage

\section*{Education}

\begin{itemize}
    \item Music Technology (BA), The University of Huddersfield, 2012.
    \begin{itemize}
    \item \emph{Final Research Project:} ``Audio Descriptor Driven
        Concatenative Synthesis of Corpus Databases'' \\
        - details of which can be found at:
        \href{http://pezz89.github.io/pysound/}{http://pezz89.github.io/pysound/}.
    \item \emph{Predicted Classification:} "Borderline 2:1/first" - refer to
        A.Harker written reference.
    \end{itemize}
  \item Music Technology BTEC Extended Diploma.
    \begin{itemize}
    \item \emph{Achievement:} Triple distinction awarded.
    \end{itemize}
\end{itemize}

\subsection*{Music Technology (BA)}
    Overview: \\
    Study involved developing a broad understanding of musical signal
    processing techniques through modules in topics such as DSP, Interactive
    Sound Design and a final research project based on a novel technique for
    audio synthesis  \\
    A detailed understanding of signal processing methods such as signal filtering,
    spectral and temporal analysis, and granular synthesis were developed
    through the practical application of these techniques for creative
    purposes. Developer environments and languages such as Matlab, Python and
    C++ were used to apply these concepts in software. An understanding of
    application in hardware was also developed through the use of
    microcontrollers to develop implementations of digital filters.


\section*{Key Skills}

\textit{Competent in the following programming languages, packages and environments:}
\begin{multicols}{3}
\begin{itemize}
    \item Python
    \item Matlab
    \item C++
    \item \LaTeX
    \item Max/MSP
    \item Vim
    \item Bash script
    \item Mac OSX
    \item Git
    \item HDF5 File system
    \item Unix
    
\end{itemize}
\end{multicols}

\section*{References}
\begin{table}[h]
\centering
\label{my-label}
\begin{tabular}{ll}
    \textit{Employer Reference} & \textit{Academic Reference} \\
    \begin{tabular}[c]{@{}l@{}}
         Axel Roebel \\
         Head of the Analysis/Synthesis Research Team \\
         IRCAM \\
         Research Institute, Paris \\
         Contact details available on request. \\
    \end{tabular} & \begin{tabular}[c]{@{}l@{}}
         Alex Harker \\
         Huddersfield University Lecturer \\
         The University of Huddersfield \\
         University Telephone: 01484 473043 \\
         E-mail: a.harker@hud.ac.uk \\
    \end{tabular}
\end{tabular}
\end{table}

% Footer
\bigskip
{\small Last updated: \today}

\end{document}
