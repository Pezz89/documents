
\documentclass{scrartcl}
\usepackage{enumitem}
\usepackage[british]{babel}
\usepackage[style=apa, backend=biber]{biblatex}
\DeclareLanguageMapping{british}{british-apa}
\usepackage{url}
\usepackage{float}
\restylefloat{table}
\usepackage{perpage}
\MakePerPage{footnote}

\addbibresource{~/PerryPerrySource/LaTeX/Hud_masters.bib}

\begin{document}

    \title{DSP Assignment 2\\Digital Audio Effects Implementation}
    \subtitle{Technical Report}
    \author{Sam Perry\\U1265119}
    \date{}
    \maketitle

    \begin{abstract}
        This report outlines the implementation, testing and evaluation of
        three real-time audio effects using the dsPIC 30F4013 digital signal
        processor.  Design choices, system specification, and final results
        are analysed.\\
        Design choices are compared to choices made in previous work to outline
        the changes required to implement such effects outside of
        unlimited resource systems.\\
        System specification is compared with alternative chips to understand
        the quality of the processor in relation to the state of DSP
        technology.\\
        Final system performance is then discussed to determine further
        changes that could be made to improve performance.
    \end{abstract}

    \section{Background/Literature}
    \subsection{Digital Signal processors Overview}
    A digital signal processor (DSP) is form of specialized microprocessor
    designed specifically for the processing of signals (such as audio signals
    in this case). When considering the quality of a DSP, there are many
    technical factors to consider, that contribute to the overall performance
    of the processor. These include:
    \begin{itemize}
        \item Clock speed
        \item Memory
        \item Bit depth
        \item Floating or fixed point calculation
    \end{itemize}

    In order to understand the specification of the dsPIC in relation to the
    standard of DSPs available, it will be compared to three other digital signal
    processors:
    \begin{itemize}
        \item Texas Instruments TMS320F2806x
        \item Freescale 56F8025
        \item Analog Devices ADSP-2126x
    \end{itemize}
    \subsection{Memory}
    \subsubsection{Flash}
    \subsubsection{RAM}
    \subsection{CPU}



    This section should include:\\
        The general requirements for digital audio processing systems, DSP
        processors\\
        dsPIC Development system\\

    \section{Design/Analysis}
    This section should include:\\
        Design of software, in Flowcode or C for artificial reverberation\\
        User Interface - Input from the switches to select Echo, Reverb, or
        Chorus effects\\
        \subsection{Echo}
        \subsection{Chorus}
        \subsection{Reverb}
    \section{Results}
    The results of real-time system implementation section should include:\\
        Implementation on dsPIC based system\\
        Testing and evaluation of the artificial reverberation system\\
        \subsection{Echo}
        \subsection{Chorus}
        \subsection{Reverb}
        
    \section{Further Work}
    The student should discuss the limitations of the system and how it could be developed further

    \section{Conclusions}
    The conclusion section should include:\\
        Critical discussion about the system, (did it work? if not, why
        not?).\\
        Statement of what the student learned from the exercise.\\

    \printbibliography

\end{document}
