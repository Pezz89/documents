\documentclass[titlepage]{scrartcl}
\usepackage{enumitem}
\usepackage[british]{babel}
\usepackage[style=apa, backend=biber]{biblatex}
\DeclareLanguageMapping{british}{british-apa}
\usepackage{url}
\usepackage{float}
\restylefloat{table}
\usepackage{perpage}
\MakePerPage{footnote}
\usepackage{abstract}
\usepackage{graphicx}
% Create hyperlinks in bibliography
\usepackage{hyperref}

\usepackage[T1]{fontenc}
\usepackage[utf8]{inputenc}
\usepackage{blindtext}
\setkomafont{disposition}{\normalfont\bfseries}


\graphicspath{
    {./resources/},
}
\addbibresource{~/PerryPerrySource/LaTeX/ExperimentalMusic_Bibliography.bib}

\newsavebox{\abstractbox}
\renewenvironment{abstract}
  {\begin{lrbox}{0}\begin{minipage}{\textwidth}
   \begin{center}\normalfont\sectfont\abstractname\end{center}\quotation}
  {\endquotation\end{minipage}\end{lrbox}%
   \global\setbox\abstractbox=\box0 }

\usepackage{etoolbox}
\makeatletter
\expandafter\patchcmd\csname\string\maketitle\endcsname
  {\vskip\z@\@plus3fill}
  {\vskip\z@\@plus2fill\box\abstractbox\vskip\z@\@plus1fill}
  {}{}
\makeatother

\DeclareCiteCommand{\citeyearpar}
    {}
    {\mkbibparens{\bibhyperref{\printdate}}}
    {\multicitedelim}
    {}

\begin{document}
    \title{Experimental Music\\Summative Assignment 2\\Essay}
    \subtitle{\LARGE{The role of electronic feedback and amplification in
    experimental music composition.}}
    \author{Sam Perry\\U1265119}
    \date{}

    \begin{abstract} 
        The use of electronic feedback as tools for musical composition has
        featured in many composition, popular for it's volatile and
        indeterminate nature.  Intrinsic to the use of feedback is the use of
        amplification, capable of artificially altering an input's energy, as a
        method for feedback control. This essay aims to provide a definition of
        these tools in their different forms, and to analyse their use in a
        range of prominent compositions. Forms of feedback will be defined,
        followed by a discussion of the musical implications of their use,
        including consideration for aspects such as process, indeterminacy,
        spectral implications and rhythmic implications. This will be related
        to a number of compositions in order to provide an overall
        understanding of their role in experimental music composition.
    \end{abstract}

    \maketitle

    \section{Defining feedback and amplification}
    A simple definition of feedback is the process of routing the output of a
    system back to the input of that system.~\parencite[p.1]{weisert2010ioi}
    This can take many form in the context of music, whether it is the acoustic
    feedback created by aiming a microphone to it's amplifier, or the digital
    feedback present in an IIR filter, in all cases a loop is created from an
    output point of a system, back to it's input.\\
    The three types of feedback to be considered are:
    \begin{itemize}
        \item{Acoustic Feedback}
        \item{Electronic Feedback}
        \item{Mathmatical Feedback}
    \end{itemize}
        
        \subsection{Acoustic Feedback}
        \subsection{Electronic Feedback}
        \subsection{Mathmatical FeedbacK}
        Rational Melody XXI - Tom Johnson
        Not electronic feedback, but serves as an example that feedback is not
        limited to electronics.
        IIR filter example?
        \subsection{Amplification}

    \section{Musical Aspects and Implications of Feedback Systems}
        \subsection{Indeterminacy}
        \subsection{Process and Control}
        \subsection{Rhythmic/Temporal Implications of Feedback}
        \subsection{Spectral Implications of Feedback}
        \subsection{Dynamic Implications of Artificial Amplitude Adjustment}

    \section{Composition Analysis}
        \subsection{Acoustic Feedback}
            Steve Reich's Pendulum Music
            Robert Ashley's The Wolfman - ref: kyle gann - robert ashley

        \subsection{Electronic Feedback}
            David Tutor's Untitled (1996), Toneburst (2004), and Pulsers (1996)
            Gordon Mumma's Hornpipe

        \subsection{Amplification}
            John Cage's Cartridge Music
            Stockhausen's Mikrophonie

    \printbibliography

\end{document}
