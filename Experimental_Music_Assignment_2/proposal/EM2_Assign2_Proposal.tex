\documentclass{scrartcl}
\usepackage{enumitem}
\usepackage[british]{babel}
\usepackage[style=apa, backend=biber]{biblatex}
\DeclareLanguageMapping{british}{british-apa}
\usepackage{mathptmx}
\addtokomafont{disposition}{\rmfamily}

\addbibresource{~/PerryPerrySource/LaTeX/ExperimentalMusic_Bibliography.bib}
\DeclareCiteCommand{\citeyearpar}
    {}
    {\mkbibparens{\bibhyperref{\printdate}}}
    {\multicitedelim}
    {}

\begin{document}
    \title{Experimental Music\\Formative Assignment 2\\Essay Proposal\\}
    \author{Sam Perry\\U1265119}
    \subtitle{The role of electronics, feedback and amplification in
    experimental music composition.\\}
    \date{}
    \maketitle

    \section{Essay Objectives}
    This essay will explore the ways in which prominent composers utilise
    electronic devices and systems to compose music; focusing primarily on
    feedback and amplification as techniques for creating and manipulating
    sounds in both the analog and digital domain. This will provide a detailed
    overview of these widely used processing techniques, and will explore the
    reasons behind their popularity amongst experimental composers. The
    advantages and disadvantages of using these techniques will be explored in
    detail with regards to established concepts, and the aesthetics of such
    effects.

    \section{Subject Rationale}
    Electronics have played a key role in the development of experimental music
    in the past fifty years and have had a dramatic effect on the ways in which
    experimental music is realised. The ability to control amplitude and cause
    feedback has influenced a number of compositions, which will be explored in
    this essay.\\
    Examples of compositions include:

    \begin{enumerate}
        \item \textbf{Acoustic Feedback}
            \begin{itemize}
                \item Steve Reich's Pendulum Music~\parencite[p.31]{reich2002wom}
                \item Robert Ashley's The Wolfman~\citeyearpar{ashley2003w}
            \end{itemize}

        \item \textbf{Electronic Feedback}
            \begin{itemize}
                \item David Tutor's Untitled ~\citeyearpar{tudor1996twfle},
                    Toneburst ~\citeyearpar{tudor2004lem}, and
                    Pulsers ~\citeyearpar{tudor1996twfle}  
                \item Gordon Mumma's Hornpipe~\citeyearpar{mumma2002lem}
            \end{itemize}

        \item \textbf{Amplification}
            \begin{itemize}
                \item John Cage's Cartridge Music~\citeyearpar{cage2013cm}
                \item Stockhausen's
                    Mikrophonie~\citeyearpar{stockhausen1995mmt}
            \end{itemize}
    \end{enumerate}

    \section{Areas of Interest}
    The areas that will be explored in detail in this essay include:

    \begin{enumerate}
        \item \textbf{Types of amplification and feedback}\\
            This will discuss the different variations of the techniques based
            on the development of technology and the implications of these
            variations.

        \begin{enumerate}[label*=\arabic*.]
            \item \textbf{Analog feedback}\\
                Referring to the use of a microphone and loudspeaker to
                generate an amplified feedback loop of sounds in the cross-over
                field between the two.~\parencite[p.185]{holmes2012eaem}

            \item \textbf{Tape feedback}\\
                Referring to the technique of recording input to a tape and
                looping over a playhead to produce repetitions in the output
                signal. Used by Robert Ashley in "The
                Wolfman".~\parencite[p.186]{holmes2012eaem}

            \item \textbf{Electronic feedback}\\
                An alternative to tape feedback which works entirely
                electronically, where a signal is ``generated within an
                electronic instrument whose design enables the recirculation of
                a signal within a closed circuit''.
                ~\parencite[p.187]{holmes2012eaem} This technique features heavily
                in David Tudor's works.
                ~\parencite{tudor1996twfle, tudor2004lem}


            \item \textbf{Feedback manipulation}\\
                The delay in time between a direct signal and a feedback signal
                allows for the manipulation of repetitions. This gives scope
                for a wide range of possible manipulations to the overall
                output and is explored in pieces such as Gordon Mumma's
                ``Hornpipe''. ~\parencite[p.390]{holmes2012eaem}

            \item \textbf{Digital feedback}\\
                With the growing use of computers for musical processing,
                feedback is possible in the digital domain, as it is in the
                analog domain. This allows DSP techniques to be applied to
                feedback loops leading to significant advancements in effects
                such as artificial reverb, giving even further scope to the
                possibilities for composers.

            \item \textbf{Artificial Amplification}\\
                Amplification is an important part of any feedback system as it
                allows for control over both initial input and the feedback
                loop. It also allows for ``small
                sounds''~\parencite[p.6]{cage2011silence} to perceived at much
                higher volumes than they naturally occur. This is explored in
                John Cage's ``Cartridge Music'' and Robert Ashley's ``The
                Wolfman''.

        \end{enumerate}
    \item \textbf{Reasons for interest in these techniques}
        \begin{enumerate}[label*=\arabic*.]
            \item \textbf{Indeterminacy}\\
                Due to the ``exponentially complex patterns of information flow
                in feedback networks''~\parencite[p.11]{weisert2010ioi},
                feedback adds an element of indeterminacy to composition. The
                build up of audio on each repetition causes varying and often
                unpredictable effects in the output
                sound.~\parencite[p.100]{nyman1999em} This will be compared to
                other techniques for introducing indeterminacy (such as John
                Cage's use of the I Ching).

            \item \textbf{Rhythmic/Temporal implications of feedback}\\
                Feedback allows for the repetition of a sound source over an
                extended period of time. This has implications rhythmically that
                can be controlled by the composer. The level of signal feedback
                will determine the decay and repetition of a signal and can
                create infinite loops of a single source sound. The
                implications of this will be discussed with reference to
                compositions such as Alvin Lucier's ``I am sitting in a
                room''.~\parencite[p.57-59, 64-68]{weisert2010ioi}


            \item \textbf{Dynamic implications of artificial amplitude
                    adjustment}\\
                Both directly and as part of a feedback system, artificial
                amplification will change the perceived level of the input
                sound. This allows composers to artificially boost or attenuate
                sound in compositions and, in conjunction with a feedback loop,
                control feedback decay. This will be discussed with reference
                to many of the compositions discussed above (this is used to
                    varying degrees across practically all previously discussed
                compositions.)


        \end{enumerate}
    \item \textbf{Forms of control}
        \begin{enumerate}[label*=\arabic*.]
            \item \textbf{Process and systems}\\
                Due to it's unpredictable nature, feedback can be difficult to
                control and can produce unexpected results. Composers have used
                a variety of methods to structure and control feedback as part
                of their compositions.~\parencite{weisert2010ioi} These
                processes will be explored to understand the different methods
                for structuring compositions when composing using feedback.
        \end{enumerate}
    \end{enumerate}

    \section{Potential Issues}
    The area of feedback in experimental music has a greater depth than
    initially anticipated (as does amplification). This essay may focus more
    heavily on feedback than amplification although both will be discussed as
    they are intrinsically linked. This may need more thought when planning the
    essay.\\
    Further refinement of areas may also be needed to focus on the most
    important aspects of the subject. This will become clear through writing
    the first draft.
    \printbibliography
\end{document}
