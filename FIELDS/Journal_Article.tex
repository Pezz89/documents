
\documentclass{scrartcl}
\usepackage{enumitem}
\usepackage[british]{babel}
\usepackage[style=apa, backend=biber]{biblatex}
\DeclareLanguageMapping{british}{british-apa}
\usepackage{url}
\usepackage{float}
\restylefloat{table}
\usepackage{perpage}
\MakePerPage{footnote}
\usepackage{abstract}
\usepackage{graphicx}
% Create hyperlinks in bibliography
\usepackage{hyperref}

\usepackage[T1]{fontenc}
\usepackage[utf8]{inputenc}
\usepackage{blindtext}
\setkomafont{disposition}{\normalfont\bfseries}


\graphicspath{
    {./resources/},
}
\addbibresource{~/PerryPerrySource/LaTeX/FYP_Bibliography.bib}


\usepackage{etoolbox}
\makeatletter
\expandafter\patchcmd\csname\string\maketitle\endcsname
  {\vskip\z@\@plus3fill}
  {\vskip\z@\@plus2fill\box\abstractbox\vskip\z@\@plus1fill}
  {}{}
\makeatother

\DeclareCiteCommand{\citeyearpar}
    {}
    {\mkbibparens{\bibhyperref{\printdate}}}
    {\multicitedelim}
    {}

\begin{document}
    \title{Descriptor Driven Concatenative Synthesis Tool}
    \subtitle{\LARGE{Abstract Draft}}
    \author{Sam Perry}
    \date{}

    \maketitle


    \begin{abstract} 
    A command-line tool is proposed for the exploration of a new form of audio
    synthesis known as ``concatenative-synthesis'': A form of synthesis that uses
    perceptual audio analyses to arrange small segments of audio based on their
    characteristics.  The tool is designed to synthesise representations of an
    input sound using a database of source sounds. This involves the
    segmentation and analysis of both the input sound and database, matching of
    input segments to their closest segment from the database, and the
    re-synthesis of the closest matches from the database to produce the final
    result.\\

    The aim was to produce a tool capable of generating high quality sonic
    representations of an input, and to present a variety of examples that
    demonstrated the breadth of possibilities that this style of synthesis has
    to offer. There are a number of other projects that use this form of
    synthesis, however this project aims primarily to explore the further
    potential offered through the offline processing of large databases, of
    which considerably less research exists.\\

    Overall, results demonstrate the wide variety of sounds that can be
    produced using this method of synthesis. A number of technical issues were
    outlined that impeded the overall quality of results and efficiency of the
    software. However, the project clearly demonstrates the strong potential
    for this type synthesis to be used for creative purposes.
    \end{abstract}

\end{document}
