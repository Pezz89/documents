
\documentclass{scrartcl}
\usepackage{enumitem}
\usepackage[british]{babel}
\usepackage[style=apa, backend=biber]{biblatex}
\DeclareLanguageMapping{british}{british-apa}
\usepackage{url}
\usepackage{float}
\restylefloat{table}
\usepackage{perpage}
\MakePerPage{footnote}
\usepackage{abstract}
\usepackage{graphicx}
% Create hyperlinks in bibliography
\usepackage{hyperref}

\usepackage[T1]{fontenc}
\usepackage[utf8]{inputenc}
\usepackage{blindtext}
\setkomafont{disposition}{\normalfont\bfseries}


\graphicspath{
    {./resources/},
}
\addbibresource{~/PerryPerrySource/LaTeX/FYP_Bibliography.bib}

\newsavebox{\abstractbox}
\renewenvironment{abstract}
  {\begin{lrbox}{0}\begin{minipage}{\textwidth}
   \begin{center}\normalfont\sectfont\abstractname\end{center}\quotation}
  {\endquotation\end{minipage}\end{lrbox}%
   \global\setbox\abstractbox=\box0 }

\usepackage{etoolbox}
\makeatletter
\expandafter\patchcmd\csname\string\maketitle\endcsname
  {\vskip\z@\@plus3fill}
  {\vskip\z@\@plus2fill\box\abstractbox\vskip\z@\@plus1fill}
  {}{}
\makeatother

\DeclareCiteCommand{\citeyearpar}
    {}
    {\mkbibparens{\bibhyperref{\printdate}}}
    {\multicitedelim}
    {}

\begin{document}
    \title{Independant Project\\Descriptor Driven Concatenative Synthesis Tool}
    \subtitle{\LARGE{Results Analysis}}
    \author{Sam Perry\\U1265119}
    \date{}

    \maketitle

    \section{Overview}
    This document aims to provide reasoning for the sonic characteristics of
    the audio produced in this project. Samples have been provided to
    demonstrate the breadth of possibilities for the concatenator script.
    Samples use a range of databases of varying sizes and aim to show both the
    strengths of this script and areas that could be developed further to
    improve results.

    \section{Sample Analysis}
    Samples are organized in folder named by their source\char`_target databases.
    
    \subsection{Guitar\char`_Guitar}
    Source Database: University of Iowa Sample Database - Guitar\\
    Target Database: David Chaplin's Guitar Samples\\

    These samples were generated with the intention of matching the clean
    guitar to a clean guitar database. It was thought that in an ideal
    situation this would produce a result similar to the original but with the
    tonal qualities of the sampled guitar. The result is far from this as the
    current descriptors do not provide a description that is sufficiently
    accurate. However, The General contour of the clean guitar can be heard and
    the ``plucky'' qualities of the sampled database are definitely present.

    \subsection{Guitar\char`_Violin}
    Source Database: University of Iowa Sample Database - Violin\\
    Target Database: David Chaplin's Guitar Samples\\
    Overall, result produced were smooth and for the majority of the samples
    used, are similar to that of a stringed instrument. The occasional glitch
    detracts from this and there are a number of reason for these small glitches.
    At around 3 seconds, a high pitched screech can be heard that does not fit
    in with it's neighbouring samples. When comparing this to the event in the
    original file, it can be seen that this is a sharp decay and pause that is
    analysed as an inharmonic signal with very different characteristics to
    it's neighbouring samples. Use of a viterbi algorithm would most likely
    help to minimise these types of error, alongside better handling of
    inharmonic samples.

    \subsection{Guitar\char`_Xylophone}
    Source Database: University of Iowa Sample Database - Xylophone\\
    Target Database: David Chaplin's Guitar Samples\\
    This example demonstrated interesting timbral qualities due to the rich
    variety of harmonic percussive sounds in the Xylophone database. By
    matching primarily on pitch and spectral content, a mixture of transient
    grains are concatenated with reasonable results.

    \subsection{Multiple\char`_Vocal}
    Source Database: Alex Harker Vocal Database\\
    Target Database: David Chaplin's Guitar Samples, Rayman Spoken Word
    Database\\
    This database provided interesting results in terms of timbre as the outcome
    is a mixture of vocal techniques. This sounds entirely synthetic which is
    expected, and there are glitches as with all the results. However, the results
    appear to be perceptually related to both the source and target databases.
    Vocal qualities can be heard through the clicks, hums and other elements
    that are combined. Continuity between grains is a significant problem as
    selected grains do not flow in the same way that natural vocals might.
    Again, comparing matches in a post matching process such as a viterbi
    decoding stage would help to counter this effect.
    A stretched version of Clean\char`_1 was included to demonstrate the results of
    setting different values for matching and synthesis overlaps. By setting the
    matching overlap to 8 and the synthesis overlap to 4, the synthesis was
    stretched by a factor of 2.

    \subsection{Spoken\char`_Xylophone}
    Source Database: University of Iowa Sample Database - Xylophone\\
    Target Database: Rayman Spoken Word Database\\

    This example was generated during development and has been included as it
    demonstrated an effective mix between the qualities of the two databases.
    The pitch and envelope of the vocals can be heard whilst the timbre of the
    Xylophone database is also clear. Glitchy pitch shifted grains can be heard
    in the background as this was a problem with the project at the time and
    has since been fixed by altering handling of inharmonic grains in the
    picth shifter.

    \subsection{Vocal\char`_Marimba}
    Source Database: University of Iowa Sample Database - Xylophone\\
    Target Database: Danny Darko Acapella Vocals Sample\\
    This sample highlights a number of problems with the current implementation
    of this algorithm.
    There is a considerable amount of noise as a result of the intensity
    scaling. This has increased quite sounds with low signal/noise ratios and
    as a result this can be heard clearly in the results.
    The clicks produced through poor performance of the pitch shifter have also
    degraded the signal during synthesis. Replacing the pitch shifter with a
    more effective algorithm would most likely improve results.

    \subsection{Vocal\char`_Wind}
    Source Database: University of Iowa Sample Database - Xylophone\\
    Target Database: Danny Darko Acapella Vocals Sample\\
    These examples are significantly less satisfying. Inharmonic noise from the
    player taking breaths mixes poorly to generated a layer of breathy noise.
    This is further emphasized by the pitch shifting of this noise. This
    demonstrates an inherent problem with pitch shifting technique used to
    transform grains, in that inharmonic audio is also transformed. This would
    be greatly improved with the use of a better pitch shifting algorithm.

    \section{Overall Performance}
    Overall, it was difficult to generate results of a high quality due to the
    current issues with the algorithm used. During development, much time was
    spent increasing the speed of the algorithm to make it possible to create a
    high enough quantity of results. On reflection, spending more time on
    improving output quality may have yielded better results. However,
    overall it is clear that this style of synthesis shows real potential for
    generating a wide range of interesting sound and as a prototype or proof of
    concept, the script achieves a reasonable quality of result.

    \section{Database Sources}
    \begin{table}[H]
    \centering
    \label{my-label}
    \begin{tabular}{ll}
        \textbf{Database Name}                       & \textbf{Accessed from}                            \\
        University of Iowa Sample Database  & http://theremin.music.uiowa.edu/MIS.html \\
        Danny Darko Acapella Vocals Sample & http://www.dannydarko.net/category/remix-stems/\\
        Alex Harker's Vocal Sample Database & Not Publicly Available                   \\
        David Chaplin's Guitar Samples      & Not Publicly Available  \\
        Rayman Spoken Word Database      & Not Publicly Available                  
    \end{tabular}
    \end{table}

\end{document}
