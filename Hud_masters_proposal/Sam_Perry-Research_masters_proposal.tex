\documentclass{scrartcl}
\usepackage{enumitem}
\usepackage[british]{babel}
\usepackage[style=apa, backend=biber]{biblatex}
\DeclareLanguageMapping{british}{british-apa}
\usepackage{url}
\usepackage{float}
\restylefloat{table}
\usepackage{perpage}
\MakePerPage{footnote}
\usepackage{graphicx}
% Create hyperlinks in bibliography
\usepackage{hyperref}

\newsavebox{\abstractbox}

\usepackage{etoolbox}
\makeatletter
\expandafter\patchcmd\csname\string\maketitle\endcsname
  {\vskip\z@\@plus3fill}
  {\vskip\z@\@plus2fill\box\abstractbox\vskip\z@\@plus1fill}
  {}{}
\makeatother

\DeclareCiteCommand{\citeyearpar}
    {}
    {\mkbibparens{\bibhyperref{\printdate}}}
    {\multicitedelim}
    {}
\usepackage[T1]{fontenc}
\usepackage[utf8]{inputenc}
\usepackage{blindtext}
\setkomafont{disposition}{\normalfont\bfseries}

\addbibresource{~/PerryPerrySource/LaTeX/Hud_masters.bib}

\begin{document}

    \title{Huddersfield Research Masters}
    \subtitle{Combined F0 Estimation Algorithm Proposal}
    \author{Sam Perry}
    \date{}
    \maketitle

    \begin{abstract}
        The pitch of audio is a perceptually important characteristic as it
        forms the building block for musical characteristics such as key,
        melody, and harmony. Many methods have been developed for estimating
        the fundamental frequency of a signal, however very few have come close
        to generating estimations that resemble human perception of pitch with
        the same level of detail. Factors such as noise and the absence of a
        clear fundamental frequency cause erroneous results in algorithms and
        the concept of polyphony further complicates the problem as this
        requires the separation of different notes. The variety of estimation
        algorithms available has lead to a selection of methods that each
        perform to varying standards depending on conditions. For example,
        time-domain approaches, such as the autocorrelation approach, are able
        to detect the correct pitch of a signal more accurately than frequency
        domain approaches, such as the harmonic-product spectrum method, when
        the fundamental frequency is missing. However neither is able to detect
        multiple pitches in the way that the MUSIC algorithm can.  
    \end{abstract}

    \section{Overview}
    This project would aim to explore the possibility of combining pre-existing
    algorithms based on audio descriptor analyses in order to adaptively select
    the algorithm with the best chance of an accurate estimate. The aim would
    be to create a robust tool for offline (and potentially realtime)
    estimation of F0 values.

    \section{Background/Existing Techniques}
    \subsection{F0 Estimation Techniques}
    The most popular F0 estimation algorithms can be categorized as one of
    two types:
    \begin{itemize}
        \item Spectral Techniques
        \item Temporal Techniques
    \end{itemize}
    \subsubsection{Spectral Techniques}
    Spectral techniques focus on analysing the spectral content of the signal
    by using output from an FFT to perform further processing to determine the
    F0 value. Types of technique that can be categorized in this way include:
    \begin{itemize}
        \item Harmonic Product Spectrum~\parencite[p.8]{smyth2015hps}
        \item Cepstral analysis
        \item Maximum likelihood
    \end{itemize}
    \subsubsection{Temporal Techniques}
    Temporal techniques attempt to calculate the periodicity of the signal.
    This can then be inverted to produce the frequency.
    Types of technique that can be categorized in this way include:
    \begin{itemize}
        \item Autocorrelation~\parencite[p.98]{lerch2012itaca}
        \item Zero-crossing~\parencite[p.98]{lerch2012itaca}
        \item NCCF (normalized cross correlation function)~\parencite{kasi2015yaapt}
    \end{itemize}
    \subsection{Current methods for technique combination} 
    \label{sec:ComMeth}
    There is considerably less research into the combination of multiple
    algorithms for the improvement of results. Limited research has been
    carried out into the effects of training supervised learning algorithms to
    pick results based on circumstances.
    The ``Yet Another Algorithm for Pitch Tracking'' algorithm attempts to
    refine temporal analysis results through the analysis of spectral
    information.~\cite{kasi2015yaapt}
    Overall there remains a large scope for the type of research proposed.

    \section{Methodology}
        A detailed analysis of a variety of the most prominent F0 estimation
        techniques will be presented, to determine the quantity of methods
        needed and which methods will produce the best quality results. Methods
        for selecting an algorithm will also require significant further
        research.  Potential techniques to be explored include: 
        \begin{itemize}
            \item Applying machine learning algorithms in order to
                automatically determine the best algorithm as described in
                section \ref{sec:ComMeth}~\parencite{bogason2015ffesl}
            \item Leveraging information gained from prior feature extraction
                to determine aspects such as the signal's noisiness in order to
                select the algorithm best suited to this description.
            \item Dynamic algorithm parameter adoption to improve the likelihood
                of an accurate estimation based on descriptors. Adapting window
                size for example.~\parencite{liuni2012aasas}
        \end{itemize}
        Having determined the optimal set of estimation algorithms and
        selection techniques, these will be implemented in python, or
        potentially a faster compiled language such as C, to create a tool
        capable of creating robust F0 estimations for a range of varying audio
        files.
        
    \section{Significance of Research}
        This research aims to explore possible improvements to the overall
        robustness and general accuracy of pitch detection and thus has
        significance in fields such as music and speech analysis and
        transformations. By taking a higher level approach to the problem, it
        is hoped that the careful combination of algorithms will yield a
        superior overall outcome to that of the individual algorithms.

    \section{Timeline}
    \begin{table}[H]
    \centering
    \label{my-label}
    \begin{tabular}{ll}
    Month 1 - 3  & Initial research into methods and combination techniques                                                                   \\
                 & as well as the set up of initial framework for code if necessary.                                                          \\
    Month 3 - 6  & Implementation and testing of individual algorithms.                                                                       \\
    Month 6 - 9  & Implementation of combination methods.                                                                                     \\
    Month 9 - 12 & Analysis of results and work on method improvements.                                                                      
    \end{tabular}
    \end{table}

    \printbibliography

\end{document}
