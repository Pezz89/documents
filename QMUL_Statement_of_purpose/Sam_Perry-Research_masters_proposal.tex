\documentclass[10pt,letterpaper]{article}

\usepackage{hyperref}
\usepackage{geometry}
\usepackage{enumitem}
\usepackage{multicol}
\newcommand{\tabitem}{~~\llap{\textbullet}~~}

% Fonts
\usepackage[T1]{fontenc}
\usepackage[urw-garamond]{mathdesign}

% Set your name here
\def\name{Samuel Perry}

% The following metadata will show up in the PDF properties
\hypersetup{
  colorlinks = true,
  urlcolor = black,
  pdfauthor = {\name},
  pdfkeywords = {DSP, Programmer},
  pdftitle = {\name: Curriculum Vitae},
  pdfsubject = {Curriculum Vitae},
  pdfpagemode = UseNone
}

\geometry{
  body={6.5in, 9.0in},
  left=1.0in,
  top=1.0in
}

% Customize page headers
\pagestyle{myheadings}
\markright{\name}
\thispagestyle{empty}

% Custom section fonts
\usepackage{sectsty}
\sectionfont{\rmfamily\mdseries\Large}
\subsectionfont{\rmfamily\mdseries\itshape\large}

% Other possible font commands include:
% \ttfamily for teletype,
% \sffamily for sans serif,
% \bfseries for bold,
% \scshape for small caps,
% \normalsize, \large, \Large, \LARGE sizes.

% Don't indent paragraphs.
\setlength\parindent{0em}

% Make lists without bullets and compact spacing
\renewenvironment{itemize}{
  \begin{list}{}{
    \setlength{\leftmargin}{1.5em}
    \setlength{\itemsep}{0.25em}
    \setlength{\parskip}{0pt}
    \setlength{\parsep}{0.25em}
  }
}{
  \end{list}
}
\setlist[enumerate]{itemsep=0.25em}

\begin{document}

% Place name at left
{\huge\name}

% Alternatively, print name centered and bold:
%\centerline{\huge \bf \name}

\bigskip

\begin{minipage}[t]{0.495\textwidth}
  20 Lower Luton Road\\
  Wheathampstead\\
  Hertfordshire\\
  AL4 8QZ\\

\end{minipage}
\begin{minipage}[t]{0.495\textwidth}
    Phone: (+44) 7766 521596\\
    Email: \href{mailto:samuel.perry89@gmail.com}{samuel.perry89@gmail.com} \\
    Linked-in: \\\href{https://uk.linkedin.com/in/sam-perry-04245438}{https://uk.linkedin.com/in/sam-perry-04245438}
\end{minipage}

\section*{\Large Sound and Music Computing MSc \\ \large Statement of Purpose}
The sound and music computing MSc offers a curriculum that is well suited to
continue my studies in the area of audio signal processing. I see the course as
an opportunity to broaden my knowledge of techniques for analysing and
synthesizing sounds digitally. This would build on my current understanding of
these techniques that has been developed over the past four years, during my
time studying at the University of Huddersfield and through working on the
Analysis and Synthesis team in the IRCAM research institute.\\ 
My time spent at the IRCAM research institute provided me with a valuable
insight into the ways that audio research is carried out and I understand that
the Centre for Digital Music carries out research of a similar nature. For
example I am already familiar with the Sonic Visualiser program which is not
dissimilar to the AudioSculpt software which was used extensively during my
internship at IRCAM. Due to the similarities between the two facilities, I
feel that the style of study on this course would be a logical step forward
from the type of work I encountered at IRCAM.\\ 
In addition to this I also have a reasonable understanding of audio descriptor
analysis techniques such as pitch and timbre analyses due to research carried
out on my final year project (see
\href{http://pezz89.github.io/pysound/index.html}{http://pezz89.github.io/pysound/index.html}
for details). This would most likely be useful prior knowledge for modules such
as the Music Analysis and Synthesis modules. \\
I would also be interested in other module available such as the machine
learning module, that would give me the opportunity to study a subject I have
basic knowledge of, but have not had the opportunity to explore in detail. I
believe this opportunity would be both interesting and useful for my future
endeavours.  I am also keen to develop my programming ability and continue
developing my knowledge of languages such as Python, Matlab and C++. Given the
technical nature of the course, I imagine that my  current knowledge of these
languages would be beneficial. \\
Studying on this course would also be an opportunity meet like minded
individuals and develop professional relationships in the industry I wish to
pursue a career in. My internship allowed me to network with a range of
researchers with a variety of specialist subjects and discuss thoughts and
ideas. I found this extremely beneficial to my understanding of this field of
research and would enjoy the oppertunity to network in a similar fashion.\\
I would expect that this course will provide the necessary skills to develop a
career in DSP engineering or provide a basis for further academic research in
these fields. On successful completion of this course I would look to either
further my studies as a PhD candidate or search for a job in commercial DSP or
general programming.\\
Overall I believe that I am a candidate that is well suited to the requirements
of this masters course. Given my previous studies and experience, I am
confident that I have the skill set and attitude required to complete a course
such as this.\\
\newline
Thank you for your consideration.

% Footer
\bigskip
{\small Last updated: \today}

\end{document}
