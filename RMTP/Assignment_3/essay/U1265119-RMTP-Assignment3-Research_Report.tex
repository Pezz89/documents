\documentclass[titlepage]{scrartcl}
\usepackage{enumitem}
\usepackage[british]{babel}
\usepackage[style=apa, backend=biber]{biblatex}
\DeclareLanguageMapping{british}{british-apa}
\usepackage{url}
\usepackage{float}
\restylefloat{table}
\usepackage{perpage}
\MakePerPage{footnote}
\usepackage{abstract}
\usepackage{graphicx}
% Create hyperlinks in bibliography
\usepackage{hyperref}
\usepackage{amsmath}

\usepackage[T1]{fontenc}
\usepackage[utf8]{inputenc}
\usepackage{blindtext}
\setkomafont{disposition}{\normalfont\bfseries}

\usepackage{pdfpages}


\graphicspath{
    {./resources/},
}
\addbibresource{~/PerryPerrySource/LaTeX/RMTP_Bibliography.bib}

\newsavebox{\abstractbox}
\renewenvironment{abstract}
  {\begin{lrbox}{0}\begin{minipage}{\textwidth}
   \begin{center}\normalfont\sectfont\abstractname\end{center}\quotation}
  {\endquotation\end{minipage}\end{lrbox}%
   \global\setbox\abstractbox=\box0 }

\usepackage{ifthen}

\newcommand*{\appendixmore}{%
  \renewcommand*{\othersectionlevelsformat}[1]{%
    \ifthenelse{\equal{##1}{section}}{\appendixname~}{}%
    \csname the##1\endcsname\autodot\enskip}
  \renewcommand*{\sectionmarkformat}{%
    \appendixname~\thesection\autodot\enskip}
}

\usepackage{etoolbox}
\makeatletter
\expandafter\patchcmd\csname\string\maketitle\endcsname
  {\vskip\z@\@plus3fill}
  {\vskip\z@\@plus2fill\box\abstractbox\vskip\z@\@plus1fill}
  {}{}
\makeatother

\DeclareCiteCommand{\citeyearpar}
    {}
    {\mkbibparens{\bibhyperref{\printdate}}}
    {\multicitedelim}
    {}

\begin{document}
    \title{Researching Music, Technology and Performance\\Summative Assignment 3\\Research Report}
    \subtitle{\LARGE{Is the Acceptability of Artificial Music Processing Dependent on Genre?}}
    \author{Sam Perry\\U1265119}
    \date{}

    \begin{abstract} 
        This report presents research into the effects of artificial electronic
        processing on musical quality. Examining data collected through the use
        of a survey, comparisons are made to observe any links between the
        increased use of artificial processing and public opinion on musical
        quality. Results presented indicate a possible link between these
        factors and suggestions are made as to possible reasons for these
        results. Limitations of the research are also acknowledged with
        suggestions for possible alterations and improvements to be considered
        in any further research.
    \end{abstract}

    \maketitle

    \section{Background}
    Since the introduction of sound recording, audio technicians have sought
    ways to produce the highest quality musical recordings possible. More
    recently, technology has been used creatively for the improvement and
    alteration of recordings.~\parencite[p.80, 258-263]{millard2005aor} A
    plethora of tools and techniques have been developed with the aim of
    improving the quality of audio; Tools such as auto-tuners, compressors,
    equalisers and artificial reverbs have been created to alter the input
    sound in an infinite number of ways.  Although this may seem like an
    obvious improvement in the production of music, it raises questions as to
    the quality and musicianship of the composers and performers. Where once an
    out of tune melody was unfixable, methods are now available for the
    specific purpose of hiding these imperfections. Room acoustics can now be
    transformed, drum rhythms tightened and dynamics altered to improve almost
    any aspect of the original recording. This potentially allows for greater
    degrees of of error from the original musician, allowing a technician to
    compensate.~\parencite{bbc2014crm}\\ 
    These techniques are used to varying degrees in different styles of music
    and are not purely used as a means for musical correction. Many musical
    genres use technology in varying aspects of their realisation.  The
    invention of instruments such as the synthesizer has lead to a variety of
    electronic based musical genres and has greatly influenced modern popular
    music. Where a traditional classical music recording may utilise modern
    production techniques to produce the most accurate representation of the
    original performance, a electronic dance music recording may synthesise a
    significant amount of sound purely through electronic means.\\
    Does this artificial synthesis and enhancement of sound detract from
    the listener's perception of the musicianship? This will be analysed to
    give an outline as to the possible relationships between these factors
    across a wide variety of musical genres.

    \section{Literature Review}\label{LR}
    Review of the literature related to this project reveals extensive research
    into subject areas closely related to this project. Research into musical
    preference has been conducted in a number of instances and demonstrates
    that there are a large number of social and psychoacoustic aspects that
    influence musical preference. There is a significant amount of
    information regarding musical talent and what defines high quality music.
    The final aspect that relates closely to this research is that of sound
    quality, which has been researched at length in order to determine not only
    the best ways of recording sound, but highly technical methods for the
    analysis and categorisations of sounds based on human
    perception~\parencite{hal}.

    \subsection{Musical Preference Research}
    There are a number of resources available that attempt to analyse some of
    the physiological effects of music. Research such as that carried out by
    Ladinig and Schellenberg attempts to analyse participant emotional
    reactions to different types of music, measuring for emotional reactions
    such as perceived intensity, happiness, and sadness~\parencite{lum}. This
    research suggests that character traits such as extroversion and
    agreeableness influence a persons preference on music. This is one of many
    studies that focus on these types of
    influence~\parencite{kessler2004semmp,eamim}
    Another key influence is that of social norms and society. Much research
    has also been undertaken in recognising relationships between styles of
    music and social phenomena. This can be found in Popular Music and
    Society~\parencite{longhurst2007pmas} that provides a substantial insight
    into the effects of society on music and vice-versa. Issues such as
    technology, sexuality and ethnicity all effect perception.
    
    \subsection{Musical Talent Research}
    An equally broad range of literature can be found regarding opinions on what
    makes a high quality composition or performance. Literature on musical
    talent outlines perceptions of what is regarded as high quality music and
    what is seen to be favourable traits in terms of composition and
    performance. This is predominantly focused on the development of musical
    talent in children, however the concepts apply to the research described in
    this report.~\parencite{spark, hoffman2015blessed, kingsbury2001mtp}

    \section{Project Aims}
    Despite the wide range of information available on related areas outlined
    in section~\ref{LR}, there is little in the area of the effects of
    electronics specifically on musical preference. The research presented in
    this report aims to provide a basis for addressing this apparent lack of
    information.

    This project aims to gather quantifiable data regarding the effects of
    artificial electronic processing on musical preference. Through the
    combined use of questionnaires, the amalgamation of information gathered
    from previous research and analysis of collected data, it is hoped that
    this report will provide an indication as to whether artificial music
    processing has any effect on opinions across a variety of popular music
    genres. 

    \section{Methodology}
    Research will be carried out through the use of a questionnaire to gather
    opinions on artificial processing in relation to a wide range of genres.
    A questionnaire has been created to collect quantitative data from anonymous
    participants. This data will then be analysed to reveal any correlations
    between views on musical quality and perceptions of artificial processing.
    These results will then be compared to previous research presented in
    section~\ref{LR} to give an overall understanding of the impact of
    artificial processing.

    \subsection{Procedure}
    The majority of original research has been carried out through the design and
    collection of data from a questionnaire. The questionnaire was designed for
    the quantitative analysis of anonymous opinions on the perceived
    artificialness and musical quality of a range of musical samples.
    Comparison of results with previous research is presented and
    observations are made on possible reasons for results.

    \subsubsection{Questionnaire Design}
    The questionnaire was designed with two main sections. The first gave an
    indication of the participant's musical knowledge and preference, the second
    collected the main data based on the participant's use of a likeart scales
    to rate aspects of the excepts.\\
    It was expected that a number of factors would influence a participant's
    rating of both musical talent and synthetic characteristics. The most
    prominent factor was thought to be the participant's background in terms of
    music and music technology training. A participant with an in depth
    understanding of composition and performance would most likely have a
    heightened sensitivity to the musical aspects of the examples that an
    untrained participant may not pick up on. To account for this, participants
    were simply asked to state if they have any musical/music technological
    training. This could then be taken into account in the analysis of results.
    Another factor that would most likely affect a participants decision was
    their preference with regards to musical style. As stated in
    section~\ref{LR}, many factors influence a person's musical preference in
    terms of genre, and so accounting for bias towards certain musical styles
    was required in an attempt to create more a more objective analysis.\\

    To begin selection of suitable material for the questionnaire, 5 genres were selected:
    \begin{itemize}
        \item Classical
        \item Rock
        \item Electronic
        \item Indie
        \item Pop
    \end{itemize}
    A selection of 40 thirty second audio samples were then used for ratings,
    sourced from a wide variety of artists across these 5 genres.
    Participants were asked to rate these samples for their musical
    accomplishment, artificialness and how much they liked the audio clip.
    Musicality and artificialness would give a clear and simple representation
    of the value placed for these parameters, which could then be compared on a
    genre by genre basis to determine any outstanding patterns. The musical
    preference parameter was added due to the wide variety of factors that
    affect musical preference. It was thought that people may value the
    musicality of pieces in genres they like more highly than genres they
    dislike. The results of these hypothesis are determined in section~\ref{analysis}\\
    
    A section for participant comments was created at the bottom of the test to
    allow participants to share thoughts on the survey. This would allow
    participants to provide further insights that might provide information that
    would not have otherwise been considered.

    \subsection{Results}
    Results were predominantly quantitative, and so did not require the same
    degree of subjective analysis that would have been the case with
    qualitative results.  A total of 20 participants took the survey, resulting
    in a dataset containing 800 data points across the range of genres.  This
    provided enough data for a reasonable analysis of comparable opinions
    across the range of participants. The amount of data collected may not be
    sufficient for conclusive evidence as to the effects of electronic
    processing on opinions of musicality, however it should provide an
    indication as to possible correlations between the two factors.
    There are also a number of factors, beyond control in the context of this
    research, that will have affected the results.\\

    The varying levels of musical training across participants has most likely
    had a significant effect on results. It was expected that participant's
    opinions on the synthetic nature of pieces would be affected by this
    factor. Students that had studied music technology would most likely have a
    more objective opinion on the artificial qualities of examples and would be
    able to pick out features that would not be possible without significant
    training in this area. This has coloured ratings of artificialness and
    should be taken into account when review results.\\

    Another key factor is the participant's subjective definitions of musical
    quality and accomplishment. This factor was made clear by the comments of
    a participant who regarded the lyrical content of the examples to be the
    determining factor. This contrasted the composition arrangement
    characteristics that were the primary focus of many musically trained
    participants.\\
    
    A factor made clear by participant comments was the limitations of
    the genres provided. Although every effort was made to provide a wide
    variety of genres, the plethora of musical genres that exist made it
    impossible to include examples for every one. Genres such as rap were not
    represented in this survey for example. This genre in particular, may have
    provided interesting results from the participant who valued lyrical
    content. A wide enough variety is thought to have been chosen that results
    will still give a reasonable insight, however this may be considered in any
    potential further research.

    \section{Analysis}\label{analysis}
    Having collected a significant quantity of both quantitative and
    qualitative data from the survey, interpretation of results could then be
    made to determine possible correlations between parameters. Initially it
    was thought that by comparing participant's rating for musical quality with
    ratings for artificialness, a relationship between these two factors may
    exist that suggests an increase in synthetic perception may relate to a
    decrease in musical quality perception. To analyse this, points were
    plotted to a scatter graph and the correlation between the two variables
    was calculated. This was initially created for all data points across all
    genres. The results can be seen in figure~\ref{agmvs}.
    \begin{figure}[H]
        \caption{Comparison of musical quality and artificialness: all genres of music}
        \makebox[\textwidth]{\includegraphics[width=0.75\textwidth]{all_genres_music_v_synth}}
        \label{agmvs}
    \end{figure}
    The same analysis was then performed on a genre by genre basis. This was to
    gain a more detailed understanding of the relationship across each genre.
    It was expected that the relationship would vary as some genres are
    typically more closely associated with electronic processing than other.
    For example, musical quality would not be negatively affected by a use of
    electronic processing in the electronic genre as electronic processing is
    inherent to the production of that style of music. However, a stronger
    relationship would be expected in classical examples as traditionally,
    classical pieces make little to no use of electronic equipment. This
    hypothesis appears to hold a degree of truth, as a negative correlation can
    be found between musical quality and artificialness as shown in
    figure~\ref{cmvs}.

    \begin{figure}[H]
        \caption{Comparison of musical quality and artificialness: classical music}
        \makebox[\textwidth]{\includegraphics[width=0.75\textwidth]{classical_music_v_synth}}
        \label{cmvs}
    \end{figure}
    The opposite is true of the electronic genre, where a minor positive
    relationship is observed as illustrated in figure~\ref{emvs}.

    \begin{figure}[H]
        \caption{Comparison of musical quality and artificialness: electronic music}
        \makebox[\textwidth]{\includegraphics[width=0.75\textwidth]{electronic_music_v_synth}}
        \label{emvs}
    \end{figure}

    When comparing variable in this manner, the coefficient of determination
    has been used to measure the confidence of the correlation. This value is
    low across all analyses and suggests a weak correlation, however this is
    expected due to the large number of other variables and the subjective
    nature of the test that affects results. However it can be observed that
    there is a much higher $R^2$ value for classical, which could be attributed
    to the natural nature of classical music.
    Graphs for all genres can be found in Appendix B.\\

    Further analyses were created, regarding the relationship between the
    participant's preference in terms of genre and their perception of musical
    quality. It was thought that participants (particularly those that had not
    had significant musical training) would most likely link musical
    accomplishment with their musical preference and thus rate examples more
    highly if they were in their preferred genre, regardless of quality. By
    analysing the frequency of high ratings in preferred genres, an indication
    of the level of bias towards favoured genres was estimated. Results of this
    analysis show that for genres such as Classical, participants who favoured
    that genre rated examples highly. This is not true of all genres as a
    more even distribution of ratings is observed in the electronic genre.
    Another key factor is the actually quality of examples. If all samples were
    perceived to be of high quality by all participants then a similar problem
    would be shown. Overall this analysis was not nearly as accurate or useful
    as the initial analysis, however it does show that there are many elements
    to be considered with this type of research and the simplicity of this
    approach could be improved through further consideration of the elements.
    Graphs of these results can be found in Appendix B.

    \section{Further Improvements and Considerations for Future Research}
    This research takes a relatively simplistic approach to a significantly
    broad area of research. There are many factors that have not been accounted
    for that have significant impact on the outcome. On reflection, the
    following improvements may allow for more robust analysis and more
    convincing results than those presented in this report:

    \paragraph{Increasing the number of possible values for rating}\mbox{}\\
    A number of comments were made, suggesting that the survey could be
    increasing the range of the scale used for ratings. Participants felt that
    five possible choices was not sufficient for accurately rating all forty
    samples. Having an increased number of points would allow for more fine
    grained choice for each of the parameters and would also offer smaller
    distinctions between ratings in the analysis stage. 

    \paragraph{A wider variety of genres/varying quality of examples}\mbox{}\\
    Effort was made to design a questionnaire that would portray a wide variety
    of examples in order to gain insight into perceptions of electronic
    processing on many genres. However, it was necessary to limit the number of
    genres in order to gain a detailed analysis of each genre included in the
    survey. This resulted in many types of music being disregarded. An increase
    in the types of music would have potentially produced a clearer overall
    picture of perceptions.\\ 
    The quality of the examples was also an issue as in some genres, as a lack
    of significantly processed examples resulted in consistently low artificial
    rating. This provided little data for comparison and thus reduced the
    quality of analyses created.

    \paragraph{Familiarity of examples}\mbox{}\\
    It was noted that some participants rated samples based on their knowledge of
    the entire song, rather than just the excerpt, due to prior knowledge of
    that particular recording. This may have affected outcomes and could be
    addressed through the use of corpus databases designed for music research,
    rather than popular music that may have been experienced before.

    \section{Conclusion}
    Overall, this research has provided a rudimentary overview of the effects
    of electronic processing on the perceptions of musical quality. In
    addition, it has suggested that opinions on the use of artificial
    processing do vary based on the style of music in question. The significant
    number of limitations due to the nature of undergraduate research prevent
    any conclusive findings. However, this could potentially form the basis for
    more in depth exploration of the effects of electronic processing on music,
    particularly as this appears to be an under-explored area.
    This research provides preliminary evidence to suggest that artificial
    processing does have some form of impact on the acceptability of music and
    suggests that this does vary to a degree based on the style of music.
    
    \printbibliography

    \appendix
    \gdef\thesection{\centerline{Appendix \Alph{section}}}
    \section{}
    The following pages show the final questionnaire used for collection of results. 
    \includepdf[pages=-]{../questionnaire/questionnaire.pdf}

    \section{}\label{ap2}
    The following graphs plot the results from the questionnaire analysis. The
    solid red line indicates the line of best fit. The dotted red lines
    indicate error boundaries.
    \begin{figure}[H]
        \caption{Comparison of musical quality and artificialness: all genres of  music}
        \makebox[\textwidth]{\includegraphics[width=0.75\textwidth]{all_genres_music_v_synth}}
    \end{figure}
    \begin{figure}[H]
        \caption{Comparison of musical quality and artificialness: rock music}
        \makebox[\textwidth]{\includegraphics[width=0.75\textwidth]{rock_music_v_synth}}
    \end{figure}
    \begin{figure}[H]
        \caption{Comparison of musical quality and artificialness: classical music}
        \makebox[\textwidth]{\includegraphics[width=0.75\textwidth]{classical_music_v_synth}}
    \end{figure}
    \begin{figure}[H]
        \caption{Comparison of musical quality and artificialness: pop music}
        \makebox[\textwidth]{\includegraphics[width=0.75\textwidth]{pop_music_v_synth}}
    \end{figure}
    \begin{figure}[H]
        \caption{Comparison of musical quality and artificialness: electronic music}
        \makebox[\textwidth]{\includegraphics[width=0.75\textwidth]{electronic_music_v_synth}}
    \end{figure}
    \begin{figure}[H]
        \caption{Comparison of musical quality and artificialness: indie music}
        \makebox[\textwidth]{\includegraphics[width=0.75\textwidth]{indie_music_v_synth}}
    \end{figure}
    \begin{figure}[H]
        \caption{Comparison of musical quality and musical preference: classical music}
        \makebox[\textwidth]{\includegraphics[width=0.75\textwidth]{classical_quality_v_preference}}
    \end{figure}
    \begin{figure}[H]
        \caption{Comparison of musical quality and musical preference: electronic music}
        \makebox[\textwidth]{\includegraphics[width=0.75\textwidth]{electronic_quality_vs_preference}}
    \end{figure}
    \begin{figure}[H]
        \caption{Comparison of musical quality and musical preference: indie music}
        \makebox[\textwidth]{\includegraphics[width=0.75\textwidth]{indie_quality_v_preference}}
    \end{figure}
    \begin{figure}[H]
        \caption{Comparison of musical quality and musical preference: pop music}
        \makebox[\textwidth]{\includegraphics[width=0.75\textwidth]{pop_quality_v_preference}}
    \end{figure}
    \begin{figure}[H]
        \caption{Comparison of musical quality and musical preference: rock music}
        \makebox[\textwidth]{\includegraphics[width=0.75\textwidth]{rock_quality_v_preference}}
    \end{figure}
\end{document}
