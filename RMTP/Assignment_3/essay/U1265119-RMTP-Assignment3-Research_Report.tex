\documentclass[titlepage]{scrartcl}
\usepackage{enumitem}
\usepackage[british]{babel}
\usepackage[style=apa, backend=biber]{biblatex}
\DeclareLanguageMapping{british}{british-apa}
\usepackage{url}
\usepackage{float}
\restylefloat{table}
\usepackage{perpage}
\MakePerPage{footnote}
\usepackage{abstract}
\usepackage{graphicx}
% Create hyperlinks in bibliography
\usepackage{hyperref}
\usepackage{amsmath}

\usepackage[T1]{fontenc}
\usepackage[utf8]{inputenc}
\usepackage{blindtext}
\setkomafont{disposition}{\normalfont\bfseries}

\usepackage{pdfpages}


\graphicspath{
    {./resources/},
}
\addbibresource{~/PerryPerrySource/LaTeX/RMTP_Bibliography.bib}

\newsavebox{\abstractbox}
\renewenvironment{abstract}
  {\begin{lrbox}{0}\begin{minipage}{\textwidth}
   \begin{center}\normalfont\sectfont\abstractname\end{center}\quotation}
  {\endquotation\end{minipage}\end{lrbox}%
   \global\setbox\abstractbox=\box0 }

\usepackage{ifthen}

\newcommand*{\appendixmore}{%
  \renewcommand*{\othersectionlevelsformat}[1]{%
    \ifthenelse{\equal{##1}{section}}{\appendixname~}{}%
    \csname the##1\endcsname\autodot\enskip}
  \renewcommand*{\sectionmarkformat}{%
    \appendixname~\thesection\autodot\enskip}
}

\usepackage{etoolbox}
\makeatletter
\expandafter\patchcmd\csname\string\maketitle\endcsname
  {\vskip\z@\@plus3fill}
  {\vskip\z@\@plus2fill\box\abstractbox\vskip\z@\@plus1fill}
  {}{}
\makeatother

\DeclareCiteCommand{\citeyearpar}
    {}
    {\mkbibparens{\bibhyperref{\printdate}}}
    {\multicitedelim}
    {}

\begin{document}
    \title{Researching Music, Technology and Performance\\Summative Assignment 3\\Research Report}
    \subtitle{\LARGE{Is the Acceptability of Artificial Music Processing Dependent on Genre?}}
    \author{Sam Perry\\U1265119}
    \date{}

    \begin{abstract} 
    \end{abstract}

    \maketitle

    \section{Background}
    Since the introduction of sound recording, audio technicians have sought
    ways to produce the highest quality musical recordings possible. More
    recently, technology has been used creatively for the improvement and
    alteration of recordings.~\parencite[p.80, 258-263]{millard2005aor} A
    plethora of tools and techniques have been developed with the aim of
    improving the quality of audio; Tools such as auto-tuners, compressors,
    equalisers and artificial reverbs have been created to alter the input
    sound in an infinite number of ways.  Although this may seem like an
    obvious improvement in the production of music, it raises questions as to
    the quality and musicianship of the composers and performers. Where once an
    out of tune melody was unfixable, methods are now available for the
    specific purpose of hiding these imperfections. Room acoustics can now be
    transformed, drum rhythms tightened and dynamics altered to improve almost
    any aspect of the original recording. This potentially allows for greater
    degrees of of error from the origonal musician, allowing a technician to
    compensate.~\parencite{bbc2014crm}\\ 
    These techniques are used to varying degrees in different styles of music
    and are not purely used as a means for musical correction. Many musical
    genres use technology in varying aspects of their realisation.  The
    invention of instruments such as the synthesizer has lead to a variety of
    electronic based musical genres and has greatly influenced modern popular
    music. Where a traditional classical music recording may utilise modern
    production techniques to produce the most accurate representation of the
    original performance, a electronic dance music recording may synthesise a
    significant amount of sound purely through electronic means~\parencite{}.\\
    Does this artificial synthesis and enhancement of sound detract from
    the listener's perception of the musicianship? This will be analysed to
    give an outline as to the possible relationships between these factors
    across a wide variety of musical genres.

    \section{Literature Review}\label{LR}
    Musical preference research - what influeces a persons perception of good
    music
    Musical talent research - what defines good musicianship?
    Historic research into reasons behind begining to creatively use technology for production.

    \section{Methodology}
    Research will be carried out through the use of a questionnaire to gather
    opinions on artificial processing in relation to a wide range of genres.
    A questionnaire has been created to collect quantitive data from annonymous
    participants. This data will then be analysed to reveal any correlations
    between view on musical quality and perceptions of artificial processing.
    These results will then be compared to previous research presented in
    section~\ref{LR} to give an overall understanding of the impact of
    artificial processing.

    \subsection{Procedure}
    The majority of original research will be carried out through the design
    and collection of data from a questionnaire. The questionnaire was designed
    for the quantitive analysis of annonymous oppinions on the perceived
    artificialness and musical quality of a range of musical samples.\\
    The questionnaire was designed with two main sections. The first gave an
    indication of the participant's musical knowlege and preference, the second
    collected the main data based on the participant's use of a likeart scales
    to rate aspects of the excepts.
    \subsection{Participants}
    \subsection{Results}

    \section{Analysis}

    \section{Conclusion}
    
    \printbibliography

    \appendix
    \gdef\thesection{\centerline{Appendix \Alph{section}}}
    \section{}
    The following pages show the final questionnaire used for collection of results. 
    \includepdf[pages=-]{../questionnaire/questionnaire.pdf}
\end{document}
